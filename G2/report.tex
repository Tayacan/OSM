\documentclass{article}
%\usepackage{msmath}
\usepackage{amssymb, amsmath}
\usepackage[pdftex]{graphicx}
\usepackage{listings}              % code insert
\usepackage{color}
\usepackage[usenames,dvipsnames,svgnames,table]{xcolor}
\usepackage{wrapfig}

\lstset{
    breaklines = true,
    numbers = left,
    stepnumber = 1,
    numberstyle=\color{black},
    showstringspaces=false,
    language=C,
    frame=rlTB,
    rulecolor= \color{blue},
    basicstyle=\scriptsize\ttfamily\color{red!80!black},
      keywordstyle=\bfseries\color{blue},
      commentstyle=\color{green!40!black},
      identifierstyle=\ttfamily\color{black},
      stringstyle=\color{yellow!65!black},
}

\newcommand{\ssection}[1]{
\addcontentsline{toc}{section}{#1}
\section*{#1}}


\title{G2 - report}
\author{Ask Neve Gamby \& Maya Saietz}

\begin{document}
\maketitle
\section{Types and functions for userland processes}

\subsection{Process control blocks}
Process control blocks are defined in \texttt{proc/process.h} as the type \texttt{process\_control\_block\_t}. The type is a struct, which contains the following members:

\begin{tabular}{|p{6cm}|p{6cm}|}
    \hline
    \texttt{char executable[MAX\_NAME\_LENGTH]} & The name of the program that this process runs. \\\hline
               \texttt{process\_state\_t state} & The state of the process (\texttt{RUNNING}, \texttt{ZOMBIE} etc.) \\\hline
                            \texttt{int retval} & Only used when the process has called \texttt{process\_finish}. Contains the return value of the process. A negative value indicates an error. \\\hline
                 \texttt{process\_id\_t parent} & The process ID of the process that spawned this one. \\\hline
\end{tabular}

\subsection{The process table}
The process table is an array of \texttt{process\_control\_block\_t}'s. The process ID of a given process is its index in the process table.

We also have a \texttt{spinlock process\_table\_lock}, which must be held whenever we try to access the process table.

\subsection{\texttt{process\_init}}
This function is called during startup. It sets the \texttt{state} variable of every \texttt{process\_control\_block\_t} in the process table to \texttt{FREE}.

We don't use the spinlock in this function, because at this point, no processes have been started or can be started.

\subsection{\texttt{process\_spawn}}
\subsection{\texttt{process\_join}}
\subsection{\texttt{process\_finish}}

\section{System calls for userland processes}
Implementing the system calls was just a matter of making thin wrappers around the functions implemented in task 1. \texttt{syscall\_exec} is a wrapper \texttt{process\_spawn}, \texttt{syscall\_join} is a wrapper for \texttt{process\_join}, and \texttt{syscall\_exit} is a wrapper for \texttt{process\_finish}.

\end{document}
