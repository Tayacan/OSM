\documentclass{article}
%\usepackage{msmath}
\usepackage{amssymb, amsmath}
\usepackage[pdftex]{graphicx}
\usepackage{listings}              % code insert
\usepackage{color}
\usepackage[usenames,dvipsnames,svgnames,table]{xcolor}
\usepackage{wrapfig}

\lstset{
    breaklines = true,
    numbers = left,
    stepnumber = 1,
    numberstyle=\color{black},
    showstringspaces=false,
    language=C,
    frame=rlTB,
    rulecolor= \color{blue},
    basicstyle=\scriptsize\ttfamily\color{red!80!black},
      keywordstyle=\bfseries\color{blue},
      commentstyle=\color{green!40!black},
      identifierstyle=\ttfamily\color{black},
      stringstyle=\color{yellow!65!black},
}

\newcommand{\ssection}[1]{
\addcontentsline{toc}{section}{#1}
\section*{#1}}


\title{G2 - report}
\author{Ask Neve Gamby \& Maya Saietz}

\begin{document}
\maketitle
\section{Types and functions for userland processes}

\subsection{Process control blocks}
Process control blocks are defined in \texttt{proc/process.h} as the type \texttt{process\_control\_block\_t}. The type is a struct, which contains the following members:

\begin{tabular}{|p{6cm}|p{6cm}|}
    \hline
    \texttt{char executable[MAX\_NAME\_LENGTH]} & The name of the program that this process runs. \\\hline
               \texttt{process\_state\_t state} & The state of the process (\texttt{RUNNING}, \texttt{ZOMBIE} etc.) \\\hline
                            \texttt{int retval} & Only used when the process has called \texttt{process\_finish}. Contains the return value of the process. A negative value indicates an error. \\\hline
                 \texttt{process\_id\_t parent} & The process ID of the process that spawned this one. \\\hline
\end{tabular}
\vspace{1cm}

The \texttt{process\_state\_t} type is an \texttt{enum} containing the following values:

\begin{tabular}{|p{3cm}|p{9cm}|}
    \hline
    \texttt{RUNNING} & An active process. This does NOT mean that the process is actually executing right now, just that there is a thread which owns it and sometimes runs it. \\\hline
       \texttt{FREE} & This process ID is free. A new process can go ahead and use it. \\\hline
     \texttt{ZOMBIE} & This process has finished running and is waiting to be joined. \\\hline
   \texttt{SLEEPING} & The thread that owns this process is sleeping. \\\hline
\end{tabular}

\subsection{The process table}
The process table is an array of \texttt{process\_control\_block\_t}'s. The process ID of a given process is its index in the process table.

We also have a \texttt{spinlock process\_table\_lock}, which must be held whenever we try to access the process table.

\subsection{\texttt{process\_init}}
This function is called during startup. It sets the \texttt{state} variable of every \texttt{process\_control\_block\_t} in the process table to \texttt{FREE}.

We don't use the spinlock in this function, because at this point, no processes have been started or can be started.

\subsection{\texttt{process\_spawn}}
\texttt{process\_spawn} first find the next free process ID by calling the function \texttt{new\_pid}. If the process table is full, it kernel panics.

Once an ID has been found, it sets the state to \texttt{RUNNING}, copies the filename to the process table entry, and starts a thread with the function \texttt{process\_exec}, and the ID as argument. All of this is of course done with \texttt{process\_table\_lock} held.

\texttt{process\_exec} is a wrapper around \texttt{process\_start}. It sets the thread's \texttt{process\_id} to the given ID, and then runs \texttt{process\_start} with the name of the executable.

\subsection{\texttt{process\_join}}
\texttt{process_join} first turn off interups, and acquaire a lock on
the process table. It then checks whether the process it tries to join
has the current process as parent, and stops returing $-1$ if it is
not, thereby indicating an error (though this would be indestinguable from the
program returning $-1$). If it stops here it will release the lock and
return the interup status to previously.

If the target process is a child we will enter a loop, where we
continuesly check whether the target process has finished and become a
zombie. If it has not, we will put the current process on sleep,
release the lock and switch to another tread. When we return from
this, we will then reacquire the lock on the process table. When the
target process terminates, we will retrive the return value, free the
targets process id, release
the lock and restore the interup status to previously, before returning
the return value.

\subsection{\texttt{process\_finish}}

\texttt{process\_finish} sets the \texttt{retval} of the current process to its argument and the state to \texttt{ZOMBIE}, and then wakes any threads that might have been sleeping on the current process. Finally, it destroys the current thread's pagetable and calls \texttt{thread\_finish}.

Because the sleep queue is used, we disable interrupts in the beginning of the function, and restore the interrupt status as soon as we've released the process table lock.

\section{System calls for userland processes}
Implementing the system calls was just a matter of making thin wrappers around the functions implemented in task 1. \texttt{syscall\_exec} is a wrapper \texttt{process\_spawn}, \texttt{syscall\_join} is a wrapper for \texttt{process\_join}, and \texttt{syscall\_exit} is a wrapper for \texttt{process\_finish}.

\end{document}
