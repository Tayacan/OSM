\documentclass{article}
%\usepackage{msmath}
\usepackage{amssymb, amsmath}
\usepackage[pdftex]{graphicx}
\usepackage{listings} % code insert
\usepackage{color}
\usepackage[usenames,dvipsnames,svgnames,table]{xcolor}
\usepackage{wrapfig}

\lstset{
    breaklines = true,
    numbers = left,
    stepnumber = 1,
    numberstyle=\color{black},
    showstringspaces=false,
    language=C,
    frame=rlTB,
    rulecolor= \color{blue},
    basicstyle=\scriptsize\ttfamily\color{red!80!black},
      keywordstyle=\bfseries\color{blue},
      commentstyle=\color{green!40!black},
      identifierstyle=\ttfamily\color{black},
      stringstyle=\color{yellow!65!black},
}

\newcommand{\ssection}[1]{
\addcontentsline{toc}{section}{#1}
\section*{#1}}


\title{G3 - report}
\author{Ask Neve Gamby \& Maya Saietz}

\begin{document}
\maketitle
%
\section{TLB exceptions}
%
We had to handle 3 exceptions: load, store, and modified TLB exceptions. For the modified exception, one would expect to send some form of error to the user or a shutdown of the thread for userland threads. Since neither is possible in the current buenos, we have opted for having the kernel panic. If the kernel tries to write to somewhere it does not have write privileges to, it is very likely to be a bug, and we therefor also kernel panics.

For both the load and store TLB exceptions, we just need to load the corresponding page into the TLB. This is handled by a separate function, that both of them calls. This function first retrieves the exception state and the thread entry of the current thread. It then determines whether the missing page is even or odd. It then performs a linear search through the pagetable based on \texttt{VPN2} (the double page address), and check whether the correct subpage is valid. If the page is valid it writes it to a random position in the TLB table, and if not it kernel panics, for the same reason as for kernel panicking on userland based modified exceptions.

Due to some extensive debuging, a decent amount of debugging code as generated, and after finishing a debuging session the debug code was disabled by use of tests on a compiler defined \texttt{DEBUG} constant.
%
\section{Dynamic allocation}
%
To implement dynamic allocation, $2$ steps were necessary: first a kernel level allocation of physical space for the pages necessary, and secondly a modification of \texttt{malloc}, \texttt{free} and \texttt{heap\_init}
%
\subsection{\texttt{syscall\_memlimit}}
%
\texttt{syscall\_memlimit} starts by getting the current process and thread entries. There are several cases. If the argument is \texttt{NULL}, it returns the current heap end, which it gets from the current process entry. If the argument is smaller than the current heap end, it returns \texttt{NULL}. If the argument is equal to the current heap end, it simply returns the argument unchanged.

Finally, if the argument is not \texttt{NULL} and is larger than the current heap end, it allocates enough pages that the argument becomes a pointer to somewhere in the last page. It does this by first calculating the number of pages needed, and then looping that many times. In each iteration, it asks for a physical page and maps it to a virtual page. If there is no free physical page, it returns \texttt{NULL}. It then sets the \texttt{heap\_end} variable of the current process entry to the last byte in the last page, and returns it.
%
\subsection{\texttt{malloc} and \texttt{free}}
%
Basic implementations of \texttt{malloc} and \texttt{free} were already provided, but these used a statically allocated byte array for the heap. This meant that the heap couldn't grow dynamically. The task was to change \texttt{malloc} and \texttt{free} to use the new \texttt{syscall\_memlimit} instead.

The existing implementation of \texttt{free} was actually fine -- there is no way to unmap pages in Buenos, so simply inserting the block in the list of free blocks was sufficient.

The only change to \texttt{malloc} is in the end of the function. Before, if there was no free block of sufficient size, \texttt{malloc} would simply give up and return \texttt{NULL}. Instead, the new version calls \texttt{syscall\_memlimit}, trying to allocate the needed block. If this succeeds, it calls itself recursively with the same argument. If not, it returns \texttt{NULL}.

The real change is actually in the function \texttt{heap\_init}. Before, \texttt{heap\_init} used the byte array \texttt{heap} to initialize \texttt{free\_list}. In the new version, \texttt{heap} is gone, and we use \texttt{syscall\_memlimit} instead. We start by setting \texttt{free\_list} to point at the first address in the potential heap. Then we call \texttt{syscall\_memlimit} with that address, so we'll get one page of memory -- a page that contains the address of \texttt{free\_list}. \texttt{syscall\_memlimit} returns the actual new heap end, which is at the end of the allocated page.

\end{document}
